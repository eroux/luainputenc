% \iffalse meta-comment
%
% Copyright (C) 2009 by Elie Roux <elie.roux@telecom-bretagne.eu>
%
% This work is under the CC0 license. As an exception, the files 
% luainputenc.sty, lutf8.def and lutf8x.def have more restrictions. 
% See these file for more details.
%
% People who helped on this package:
%   Javier Bezos
%   Will Robertson
%   Hans Hagen
%   Heiko Oberdiek
%   Javier Mugica
%
% This work consists of the main source file luainputenc.dtx
% and the derived files
%   luainputenc.sty, luainputenc.lua, lutf8.def, lutf8x.def,
%   and luainputenc.pdf
%
% Unpacking:
%    tex luainputenc.dtx
% Documentation: 
%    pdflatex luainputenc.dtx
%
%    The class ltxdoc loads the configuration file ltxdoc.cfg
%    if available. Here you can specify further options, e.g.
%    use A4 as paper format:
%       \PassOptionsToClass{a4paper}{article}
%
%<*ignore>
\begingroup
  \def\x{LaTeX2e}%
\expandafter\endgroup
\ifcase 0\ifx\install y1\fi\expandafter
         \ifx\csname processbatchFile\endcsname\relax\else1\fi
         \ifx\fmtname\x\else 1\fi\relax
\else\csname fi\endcsname
%</ignore>
%<*install>
\input docstrip.tex
\Msg{************************************************************************}
\Msg{* Installation}
\Msg{* Package: luainputenc 2010/05/10 v0.97 inputenc package for LuaTeX}
\Msg{************************************************************************}

\keepsilent
\askforoverwritefalse

\let\MetaPrefix\relax

\preamble
This is a generated file.

Copyright (C) 2009 by Elie Roux <elie.roux@telecom-bretagne.eu>

This work is under the CC0 license. As an exception, the files 
luainputenc.sty, lutf8.def and lutf8x.def have more restrictions. 
See these file for more details.

People who helped on this package:
  Javier Bezos
  Will Robertson
  Hans Hagen
  Heiko Oberdiek
  Javier Mugica

This work consists of the main source file luainputenc.dtx
and the derived files
   luainputenc.sty, luainputenc.lua, lutf8.def, lutf8x.def,
   and luainputenc.pdf

\endpreamble

\let\MetaPrefix\DoubleperCent

\generate{%
  \usedir{tex/lualatex/luainputenc}%
  \file{luainputenc.sty}{\from{luainputenc.dtx}{package}}%
  \file{lutf8.def}{\from{luainputenc.dtx}{def}}%
  \file{lutf8x.def}{\from{luainputenc.dtx}{defx}}%
}

% The following hacks are to generate a lua file with lua comments starting by
% -- instead of %%

\def\MetaPrefix{-- }

\def\luapostamble{%
  \MetaPrefix^^J%
  \MetaPrefix\space End of File `\outFileName'.%
}

\def\currentpostamble{\luapostamble}%

\generate{%
  \usedir{tex/lualatex/luainputenc}%
  \file{luainputenc.lua}{\from{luainputenc.dtx}{lua}}%
}

\nopreamble
\nopostamble

\generate{%
  \file{inputenc.sty.diff}{\from{luainputenc.dtx}{patch}}%
}

\obeyspaces
\Msg{************************************************************************}
\Msg{*}
\Msg{* To finish the installation you have to move the following}
\Msg{* files into a directory searched by TeX:}
\Msg{*}
\Msg{*     luainputenc.sty luainputenc.lua lutf8.def lutf8x.def}
\Msg{*}
\Msg{* Happy TeXing!}
\Msg{*}
\Msg{************************************************************************}

\endbatchfile
%</install>
%<*ignore>
\fi
%</ignore>
% \iffalse
%<package>
%<package>\NeedsTeXFormat{LaTeX2e}
%<package>\ProvidesPackage{luainputenc}
%<package>  [2010/05/10 v0.97 inputenc package for LuaTeX]
%<package>
% \fi
%<*driver>
\documentclass{ltxdoc}
\usepackage[lmargin=5cm, textwidth=14cm]{geometry}
\EnableCrossrefs
\CodelineIndex
\begin{document}
  \DocInput{luainputenc.dtx}%
\end{document}
%</driver>
% \fi
%
% \CheckSum{0}
%
% \CharacterTable
%  {Upper-case    \A\B\C\D\E\F\G\H\I\J\K\L\M\N\O\P\Q\R\S\T\U\V\W\X\Y\Z
%   Lower-case    \a\b\c\d\e\f\g\h\i\j\k\l\m\n\o\p\q\r\s\t\u\v\w\x\y\z
%   Digits        \0\1\2\3\4\5\6\7\8\9
%   Exclamation   \!     Double quote  \"     Hash (number) \#
%   Dollar        \$     Percent       \%     Ampersand     \&
%   Acute accent  \'     Left paren    \(     Right paren   \)
%   Asterisk      \*     Plus          \+     Comma         \,
%   Minus         \-     Point         \.     Solidus       \/
%   Colon         \:     Semicolon     \;     Less than     \<
%   Equals        \=     Greater than  \>     Question mark \?
%   Commercial at \@     Left bracket  \[     Backslash     \\
%   Right bracket \]     Circumflex    \^     Underscore    \_
%   Grave accent  \`     Left brace    \{     Vertical bar  \|
%   Right brace   \}     Tilde         \~}
%
% \GetFileInfo{luainputenc.sty}
%
% \title{The \textsf{luainputenc} package}
% \date{2010/05/10 v0.97}
% \author{Elie Roux \\ \texttt{elie.roux@telecom-bretagne.eu}}
%
% \maketitle
%
% \begin{abstract}
% Input encoding management for Lua\TeX, needed only for compatibility with
% old documents. For new documents, using UTF-8 encoding and Unicode fonts is
% \emph{strongly} recommended. You've been warned!
% \end{abstract}
%
% \tableofcontents
%
% \section{Overview}
%
% From the user point of view, adapting an old document for \LuaTeX\ is really
% easy: replacing \pf{inputenc} by \pf{luainputenc} in the preamble is
% enough.
%
% \textbf{Warning:} the current version of \pf{luainputenc} automatically
% loads the \LuaTeX tra packages. It won't necessarily be the case in the
% future. So if you use some functions of \LuaTeX tra (like
% \verb+\newluaattribute+), it is advised to also explicitely use
% \pf{luatextra}. A typical part of \LaTeX\ preamble for \LuaTeX\ will be
% for example:
% 
% \begin{verbatim}
% \usepackage[utf8]{luainputenc}
% \usepackage{luatextra}
% \end{verbatim}
% 
% Another important thing to know, for \LaTeX , is that loading
% \pf{luainputenc} is necessary if you use old fonts. One could believe that
% \LuaTeX\ reading UTF-8 natively, \pf{inputenc} is not necessary anymore.
% It's not the case! The exception to this is the case where you only use OTF
% fonts. But you can't simply suppress \pf{inputenc} for your old documents.
% 
% Note that \pf{luainputenc} automatically loads \pf{inputenc} if called
% with an old engine, so you will still be able to compile your documents with
% pdf\TeX\ without changing them.
% 
% A new font encoding has been added: \code{EU2}. It is meant to be used with
% OTF fonts. It is very minimal and just defines a default font (lmodern). Use
% it if you use only OTF fonts. To use it simply call \pf{fontenc} with the
% option \code{EU2}.
% 
% EU2 encoding is special as it needs non-ASCII characters to be non-active
% (unlike other font encodings), so you cannot mix old encodings and EU2. If you
% want to use both old fonts and OTF fonts in your document, the way to do so is
% by using a new option of \pf{luainputenc}: \code{lutf8x}. This option
% overrides \LaTeX 's mechanism to change the font encoding and activates or
% unactivates non-ASCII characters. If you use this option, you will be able to
% change the encoding, for example:
% 
% \begin{verbatim}
% abc
% {
% \fontencoding{EU2}\usefont
% \font\foo="MyOtfFont.otf"\foo
% abc
% }
% abc
% \end{verbatim}
% 
% Another new option has been added to \pf{luainputenc}:
% \code{unactivate}. With this option, non-ASCII characters won't be
% activated, so you will be able to use only OTF fonts. Typically you can use
% this options with the \code{EU2} \pf{fontenc} option.
% 
%
% \section{Documentation}
%
% \subsection{Introduction}
%
% One the the most interesting new features of Lua\TeX\ is the fact that it is
% (like Omega/Aleph) not limited to 256 characters, and can now understand
% Unicode. The problem is that it does not read input the way older engines
% (like pdf\TeX) do, and thus \textsf{inputenc} is totally broken with Lua\TeX
% . This package aims at replacing \textsf{inputenc} for Lua\TeX , by adapting
% the way Lua\TeX\ handles input, and the way \textsf{inputenc} handles UTF-8.
% This package has two very distinct modes: 8-bit and UTF-8.
%
% \subsection{Overview of 8-bit mode}
%
% This package \textbf{does not} map 8-bit encodings to utf8. It allows
% Lua\TeX\ to read 8-bit characters, by converting each byte into a unicode
% character with the same character number. The resulting unicode characters
% are not true UTF-8, they are what we will call ``fake UTF-8". For example
% the byte 225 will be converted into the unicode character with number 225
% (two bytes long). It will be true UTF-8 only if the encoding is latin1.
%
% Here is how it works: the 8-bit encodings are converted into fake UTF-8, so
% that the corresponding tokens are chars with the good numbers. Then (like
% \textsf{inputenc}) it reads the char numbers, and converts it into LICR
% (\LaTeX\ Internal Character Representation), with the font encoding.
%
% In Lua\TeX\ version 0.43, a new callback called |process_output_buffer|,
% this callbacks allows to make Lua\TeX\ write 8-bit instead of UTF-8, so the
% behaviour is the same as pdfTeX as this level. For versions prior to 0.43
% though, we need to do more tricky things, described in the next paragraph.
% This machinery is disabled for Lua\TeX\ version 0.43 and superior, so you
% can keep the default behaviour, which will be compatible with pdfTeX in 
% most cases, but you can consider the machinery obsolete.
%
% For these old versions, \textsf{luainputenc} only changes the input behaviour, it does not change
% the ouput behaviour (when files are written for example). The consequence is
% that files will still be written by Lua\TeX\ in UTF-8 (fake UTF-8 in this
% case), even if the asked input encoding is a 8-bit encoding. In most cases
% it's not a problem, as most files will be written in LICR, meaning ASCII,
% which is both 8-bit and UTF-8. The problem comes when characters with a
% number $>$ 128 are written in a 8-bit encoding. This may happen if you use
% |\protect| in a section for example. In these cases, Lua\TeX\ will write
% fake UTF-8, and try to read 8-bit encoding, so it will get confused.
%
% The proposed solution is to unactivate the input conversion when we read
% certain files or extentions. This package should work with no change for
% most documents, but if you cook your own aux files with an unknown
% extention, you may have to force the package to read some files in UTF-8
% instead of 8-bit. See comments in the \texttt{.sty} file to know the useful
% commands.
%
% \subsection{Overview of UTF-8 mode}
%
% The behaviour of \textsf{inputenc} in utf8 mode is to read the input byte by
% byte, and decide if the character we are in is 1, 2, 3 or 4 bytes long, and
% then read other bytes accordingly. This behaviour fails with Lua\TeX\
% because it reads input character by character (characters do not have a
% fixed number of bytes in unicode). The result is thus an error.
%
% All characters recognized by \TeX\ are active characters, that correspond to
% a LICR macro. Then \textsf{inputenc} reads the \texttt{*.dfu} files that
% contain the correspondance between these LICR macros and a character number
% in the fonts for different font encodings (T1, OT1, etc.).
%
% \subsubsection{legacy mode}
%
% \textsf{luainputenc} can get this behaviour (we will call it \emph{legacy
% mode}, but another difference implied by the fact that Lua\TeX\ can read
% more than 256 characters is that fonts can also have more than 256
% characters. Lua\TeX\ can thus read unicode fonts. If we want to use unicode
% fonts (OTF for example), we can't use the \emph{legacy mode} anymore, as it
% would mean that we would have to rewrite a specially long
% \texttt{unicode.dfu} file, and it would be totally inefficient, as for
% instance \texttt{\'e} (unicode character number 233) would be mapped to
% \texttt{\string\'e}, and then mapped back to \texttt{\string\char\ 233}.
%
% \subsubsection{unicode font mode}
%
% To fix this, the most simple solution is to desactivate all activated
% characters, thus typing \texttt{\'e} will directly call
% \texttt{\string\char\ 233} in the unicode fonts, and produce a \texttt{\'e}.
% We will call this behaviour the \emph{unicode font mode}. To enable this
% mode, you can use the option \texttt{unactivate} in \textsf{luainputenc},
% and you must use the font encoding \texttt{EU2} provided by the \textsf{euenc}
% package. See documentation of \textsf{euenc} package for more details about
% \texttt{EU2}. To use this mode with \texttt{EU2}, you must be able to open OTF
% fonts. A simple way to do so it by using the package \textsf{luaotfload}.
%
% \subsubsection{mixed mode}
%
% But the \emph{unicode font mode} has a strong limitation (that will
% certainly dissapear with time): it cannot use non-unicode fonts. If you want
% to mix unicode fonts and old fonts, you'll have to use the \emph{mixed
% mode}. In this mode you can type some parts of your document in \emph{legacy
% mode} and some in \emph{unicode font mode}. The reason why we chose not to
% integrate this choice in the \emph{legacy mode} is that we wanted to have a
% mode that preserved most of the backward compatibility, to safely compile
% old documents; the \emph{mixed mode} introduces new things that may break
% old documents. To get the \emph{mixed mode}, you must pass the option
% \texttt{lutf8x} to \textsf{luainputenc}. This mode is the most experimental.
%
% \section{Accessing the encoding in lua}
%
% In order to access the encoding and the package option in lua, two variables
% are set: |luainputenc.package_option| contains the option passed to the
% package, and |luainputenc.encoding| that contains the encoding (defaults to
% utf8, and is utf8 even with the options |unactivate|, |utf8x|, etc.).
%
% \section{Files}
%
% This package contains a \texttt{.sty} file for both \LaTeX\ and Plain, a
% patch for inputenc to use \textsf{luainputenc} so that you can process old
% documents without changing anything, and the lua functions.
%
% \subsection{\texttt{inputenc.sty} patch}
%
% A good thing would be to patch \textsf{inputenc} to load
% \textsf{luainputenc} instead, so that you don't have to change your
% documents to load \textsf{luainputenc} especially. The \LaTeX\ team is
% extremely conservative and does not want this patch applied (maybe we will
% find a solution later). Here is a patch for inputenc.sty:
%
% \iffalse
%<*patch>
% \fi
%
%    \begin{macrocode}

   \ifnum\@tempcnta<`#2\relax
      \advance\@tempcnta\@ne
   \repeat}
+
+\begingroup\expandafter\expandafter\expandafter\endgroup   
+\expandafter\ifx\csname XeTeXversion\endcsname\relax\else
+  \RequirePackage{xetex-inputenc}
+  \DeclareOption*{\PassOptionsToPackage{\CurrentOption}{xetex-inputenc}}
+  \ProcessOptions*
+  \expandafter\endinput
+\fi
+\begingroup\expandafter\expandafter\expandafter\endgroup   
+\expandafter\ifx\csname directlua\endcsname\relax\else
+  \RequirePackage{luainputenc}
+  \DeclareOption*{\PassOptionsToPackage{\CurrentOption}{luainputenc}}
+  \ProcessOptions*
+  \expandafter\endinput
+\fi
+
 \ProcessOptions
 \endinput
 %%

%    \end{macrocode}
%
% \iffalse
%</patch>
% \fi
%
% \subsection{\texttt{luainputenc.sty}}
%
%    This file has some code from \texttt{inputenc.sty}, but also provides new
%    options, and new macros to convert from 8-bit to fake UTF-8.
%
% \iffalse
%<*package>
% \fi
%
%    \begin{macrocode}
%
%% This file was adapted from inputenc.sty, which copyright is:
%% Copyright 1993 1994 1995 1996 1997 1998 1999 2000 2001 2002 2003 2004
%% 2005 2006 The LaTeX3 Project.
%%
%% inputenc.sty is under the lppl version 1.3c or later, and can be
%% found in the base LaTeX system.
%%
%% The lppl can be found at http://www.latex-project.org/lppl.txt
%%
%% The changes to inputenc.sty are Copyright 2009 Elie Roux, and are
%% under the CC0 license.
%%
%% The changes are LuaTeX support.
%%
%% This file is distributed under the CC0 license, with clause 6 of the
%% lppl as additional restrictions.

%    \end{macrocode}
%
%    First we check if we are called with Lua\TeX , (pdf)\TeX or Xe\TeX . If
%    we are called with pdf\TeX , we default to \textsf{inputenc}, and to
%    \textsf{xetex-inputenc} if we are called with Xe\TeX . We also remap the
%    new options to \texttt{utf8} in these cases.
%
%    \begin{macrocode}

\RequirePackage{ifluatex}
\RequirePackage{ifxetex}

\ifxetex
  \RequirePackage{xetex-inputenc}
  \DeclareOption{unactivate}{\PassOptionsToPackage{utf8}{xetex-inputenc}}
  \DeclareOption{lutf8}{\PassOptionsToPackage{utf8}{xetex-inputenc}}
  \DeclareOption{lutf8x}{\PassOptionsToPackage{utf8}{xetex-inputenc}}
  \DeclareOption*{\PassOptionsToPackage{\CurrentOption}{xetex-inputenc}}
  \ProcessOptions*
  \expandafter\endinput
\fi

\ifluatex\else
  \RequirePackage{inputenc}
  \DeclareOption{unactivate}{\PassOptionsToPackage{utf8}{inputenc}}
  \DeclareOption{lutf8}{\PassOptionsToPackage{utf8}{inputenc}}
  \DeclareOption{lutf8x}{\PassOptionsToPackage{utf8}{inputenc}}
  \DeclareOption*{\PassOptionsToPackage{\CurrentOption}{inputenc}}
  \ProcessOptions*
  \expandafter\endinput
\fi

%    \end{macrocode}
%
%    Here we know we are called with Lua\TeX . We first require
%    \textsf{luatextra}, then we load the \texttt{lua} file.
%
%    \begin{macrocode}

\RequirePackage{luatextra}

\luatexUseModule{luainputenc}

%    \end{macrocode}
%
%    Here is some code from \textsf{inputenc}.
%
%    \begin{macrocode}

\def\DeclareInputMath#1{%
   \@inpenc@test
   \bgroup
      \uccode`\~#1%
      \uppercase{%
   \egroup
      \def~%
   }%
}
\def\DeclareInputText#1#2{%
   \def\reserved@a##1 ${}%
   \def\reserved@b{#2}%
   \ifcat_\expandafter\reserved@a\meaning\reserved@b$ $_%
      \DeclareInputMath{#1}{#2}%
   \else
      \DeclareInputMath{#1}{\IeC{#2}}%
   \fi
}
\def\IeC{%
  \ifx\protect\@typeset@protect
    \expandafter\@firstofone
  \else
    \noexpand\IeC
  \fi
}
%    \end{macrocode}
%
%    We changed a little the behaviour of this macro: we removed
%    |\@inpenc@loop\^^?\^^ff|, because it made no sense in UTF-8 mode. We will
%    call this line for 8-bit encodings.
%
%    Note that the code has been changed for |\endlinechar|, because in new
%    versions (from v0.43) of Lua\TeX{} the value cannot exceed 127.
%    Thus, with the old version of \textsf{luainputenc}, when trying
%    to add 10000, it fails silently, and when 10000 is substracted, the
%    new value is -1, resulting in no end of lines at all in the document.
%
%    \begin{macrocode}

\def\inputencoding#1{%
  \the\inpenc@prehook
  \gdef\@inpenc@test{\global\let\@inpenc@test\relax}%
  \edef\@inpenc@undefined{\noexpand\@inpenc@undefined@{#1}}%
  \edef\inputencodingname{#1}%
  \@inpenc@loop\^^A\^^H%
  \@inpenc@loop\^^K\^^K%
  \@inpenc@loop\^^N\^^_%
  \xdef\saved@endlinechar{\the\endlinechar }%
  \endlinechar=-1
  \xdef\saved@space@catcode{\the\catcode`\ }%
  \catcode`\ 9\relax
  \input{#1.def}%
  \endlinechar=\saved@endlinechar{}%
  \catcode`\ \saved@space@catcode\relax
  \ifx\@inpenc@test\relax\else
    \PackageWarning{inputenc}%
             {No characters defined\MessageBreak
              by input encoding change to `#1'\MessageBreak}%
  \fi
  \the\inpenc@posthook
  \luadirect{luainputenc.set_option([[#1]])}
}
\newtoks\inpenc@prehook
\newtoks\inpenc@posthook
\def\@inpenc@undefined@#1{\PackageError{inputenc}%
        {Keyboard character used is undefined\MessageBreak
         in inputencoding `#1'}%
       {You need to provide a  definition with
        \noexpand\DeclareInputText\MessageBreak or
        \noexpand\DeclareInputMath before using this key.}}%
\def\@inpenc@loop#1#2{%
  \@tempcnta`#1\relax
  \loop
    \catcode\@tempcnta\active
    \bgroup
      \uccode`\~\@tempcnta
      \uppercase{%
    \egroup
         \let~\inpenc@undefined
      }%
  \ifnum\@tempcnta<`#2\relax
     \advance\@tempcnta\@ne
  \repeat}

%    \end{macrocode}
%
%    Here we declare our options. Note that we remap \texttt{utf8} to
%    \texttt{lutf8}, because we use out \texttt{lutf8.def} instead of
%    \textsf{inputenc}'s \texttt{utf8.def}.
%
%    \begin{macrocode}

\DeclareOption{utf8}{%
  \inputencoding{lutf8}%
}

\DeclareOption{lutf8}{%
  \inputencoding{lutf8}%
}

\DeclareOption{utf8x}{%
  \inputencoding{lutf8}%
}

\DeclareOption{lutf8x}{%
  \inputencoding{lutf8x}%
}

%    \end{macrocode}
%
%    For the \texttt{unactivate} option, for \emph{unicode font mode}, we just
%    don't do anything.
%
%    \begin{macrocode}

\DeclareOption{unactivate}{%
  \edef\inputencodingname{unactivate}%
  \luadirect{luainputenc.set_option([[unactivate]])}
}

%    \end{macrocode}
%
%    All other options are 8-bit encodings, so we activate the translation
%    into fake UTF-8, and we execute the loop we removes from
%    |\inputencoding|.
%
%    \begin{macrocode}

\DeclareOption*{%
  \lIE@activate %
  \@inpenc@loop\^^?\^^ff%
  \inputencoding{\CurrentOption}%
}

%    \end{macrocode}
%
%    The rest of the file is only the machinery for LuaTeX versions without
%    the callback |process_output_buffer|, so it will be deprecated after
%    TeXLive 2009, you are not advised to use it.
%
%    \begin{macrocode}

\ifnum\luatexversion>42

    \newcommand*{\lIE@activate}[0]{%
      \luadirect{luainputenc.register_callbacks()}%
    }

\else

%    \end{macrocode}
%
%    |\lIE@setstarted| and |\lIE@setstopped| are called when the fake UTF-8
%    translation must be activated or desactivated. You can call them several
%    successive times. They are called very often, even if the package is not
%    activated (for example if it's loaded with the utf8 option), but they act
%    only if the package is activated.
%
%    \begin{macrocode}

\newcommand*\lIE@setstarted[0]{%
  \ifnum\lIE@activated=1 %
    \luadirect{luainputenc.setstarted()}%
  \fi %
}

\newcommand*\lIE@setstopped[0]{%
  \ifnum\lIE@activated=1 %
    \luadirect{luainputenc.setstopped()}%
  \fi %
}

%    \end{macrocode}
%
%    The following 5 macros are made to declare a file that will have to be
%    read in fake UTF-8 and not in 8-bit. These files are the ones that will
%    be generated by \TeX . In \textbf{no way} this means you can include true
%    UTF-8 files, it means that you can include files that have been written
%    by Lua\TeX\ with \textsf{luainputenc}, which means files in fake UTF-8.
%    The macros are very simple, when you call them with a file name (the same
%    as the one you will use with \string\input ), it will read it with or
%    without the fake UTF-8 translation. This package includes a whole bunch
%    of extentions that will be read in fake UTF-8, so the occasions to use
%    these macros will be rare, but if you use them, please report it to the
%    package maintainer.
%
%    \begin{macro}{\lIE@SetUtfFile}
%
%    If you call this macro with a file name, each time you will input this
%    file, it will be read in fake UTF-8. You can call it with a file that you
%    generate with Lua\TeX\ and that you want to include.
%
%    \begin{macrocode}

\newcommand*\lIE@SetUtfFile[1]{%
  \luadirect{luainputenc.set_unicode_file([[#1]])}%
}

%    \end{macrocode}
%
%    \end{macro}
%
%    \begin{macro}{\lIE@SetNonUtfFile}
%
%    Same as the previous macro, except that the file will be read as 8-bit.
%    This macro is useful if there is an exception in an extention (see
%    further comments).
%
%    \begin{macrocode}

\newcommand*\lIE@SetNonUtfFile[1]{%
  \luadirect{luainputenc.set_non_unicode_file([[#1]])}%
}

%    \end{macrocode}
%
%    \end{macro}
%
%    \begin{macro}{\lIE@UnsetFile}
%
%    This macro gives a file the default behaviour of its extention.
%
%    \begin{macrocode}

\newcommand*\lIE@UnsetFile[1]{%
  \luadirect{luainputenc.unset_file([[#1]])}%
}

%    \end{macrocode}
%
%    \end{macro}
%
%    \begin{macro}{\lIE@SetUtfExt}
%
%    You can tell \textsf{luainputenc} to treat all files with a particular
%    extention in a certain way. The way the file extention is checked is to
%    compare the four last characters of the filename. So if your extention
%    has only three letters, you must include the preceding dot. This macro
%    tells \textsf{luainputenc} to read all files from an extention in fake
%    UTF-8.
%
%    \begin{macrocode}

\newcommand*\lIE@SetUtfExt[1]{%
  \luadirect{luainputenc.set_unicode_extention([[#1]])}%
}

%    \end{macrocode}
%
%    \end{macro}
%
%    \begin{macro}{\lIE@SetUtfExt}
%
%    Same as before, but the files will be read in 8-bit.
%
%    \begin{macrocode}

\newcommand*\lIE@SetNonUtfExt[1]{
  \luadirect{luainputenc.set_non_unicode_extention([[#1]])}
}

%    \end{macrocode}
%
%    \end{macro}
%
%    \begin{macro}{\lIE@InputUtfFile}
%
%    This macro inputs a file in fake UTF-8. It has the ``feature" to unset
%    the behaviour on the file you will call, so to be safe, you must call
%    them with files for which the behaviour has not been set.
%
%    \begin{macrocode}


\newcommand*\lIE@InputUtfFile[1]{%
  \lIE@SetUtfFile{#1}%
  \input #1%
  \lIE@UnsetFile{#1}%
}

%    \end{macrocode}
%
%    \end{macro}
%
%    \begin{macro}{\lIE@InputNonUtfFile}
%
%    Same as before, but to read a file as 8-bit.
%
%    \begin{macrocode}

\newcommand*\lIE@InputNonUtfFile[1]{%
  \lIE@SetNonUtfFile{#1}%
  \input #1%
  \lIE@UnsetFile{#1}%
}

%    \end{macrocode}
%
%    \end{macro}
%
%    Two definitions to put the previous two macros in the user space.
%
%    \begin{macrocode}

\newcommand*\InputUtfFile[1]{%
  \lIE@InputUtfFile{#1}%
}

\newcommand*\InputNonUtfFile[1]{%
  \lIE@InputNonUtfFile{#1}%
}

\newcount\lIE@activated

\newcommand*{\lIE@activate}[0]{%
  \lIE@activated=1 %
  \lIE@setstarted %
}

\newcommand*{\lIE@FromInputenc}[1]{%
  \ifnum\lIE@activated=0 %
    \lIE@activate %
  \fi%
}

\fi

\ProcessOptions*

%    \end{macrocode}
%
% \iffalse
%</package>
% \fi
%
% \iffalse
%<*def>
% \fi
%
% \subsection{\texttt{lutf8.def}}
%
%    \begin{macrocode}
%% This file was adapted from utf8.def, which copyright is:
%% Copyright 1993 1994 1995 1996 1997 1998 1999 2000 2001 2002 2003
%% 2004 2005 2006 The LaTeX3 Project.
%%
%% utf8.def is under the lppl version 1.3c or later, and can be found
%% in the base LaTeX system.
%%
%% The lppl can be found at http://www.latex-project.org/lppl.txt
%%
%% The changes to utf8.def are Copyright 2009 Elie Roux, and are under
%% the CC0 license.
%%
%% The changes are LuaTeX support.
%%
%% This file is distributed under the CC0 license, with clause 6 of the
%% lppl as additional restrictions.

%    \end{macrocode}
%
%    Most of the file is taken from \texttt{utf8.def}, the main changes are
%    commented. A lot of code was removed, especially the codes that analysed
%    the unicode characters byte by byte.
%
%    \begin{macrocode}


\ProvidesFile{lutf8.def}
   [2010/05/10 v0.97 UTF-8 support for luainputenc]

\makeatletter
\catcode`\ \saved@space@catcode

\@inpenc@test

\ifx\@begindocumenthook\@undefined
  \makeatother
  \endinput \fi

%    \end{macrocode}
%
%    This function is changed a lot. Its aim is to map the character (first
%    argument) to a macro (second argument). In \texttt{utf8.def} it was
%    complicated as unicode was analyzed byte by byte. With Lua\TeX\ it is
%    extremely simple, we just have to activate the character, and call a
%    traditional |\DeclareInputTeXt|.
%
%    \begin{macrocode}

\gdef\DeclareUnicodeCharacter#1#2{%
 \@tempcnta"#1%
 \catcode\@tempcnta\active %
 \DeclareInputText{\the\@tempcnta}{#2}%
}

\@onlypreamble\DeclareUnicodeCharacter

\def\cdp@elt#1#2#3#4{%
  \wlog{Now handling font encoding #1 ...}%
  \lowercase{%
      \InputIfFileExists{#1enc.dfu}}%
         {\wlog{... processing UTF-8 mapping file for font encoding
                #1}%
          \catcode`\ 9\relax}%
        {\wlog{... no UTF-8 mapping file for font encoding #1}}%
}
\cdp@list

\def\DeclareFontEncoding@#1#2#3{%
  \expandafter %
  \ifx\csname T@#1\endcsname\relax %
    \def\cdp@elt{\noexpand\cdp@elt}%
    \xdef\cdp@list{\cdp@list\cdp@elt{#1}%
                    {\default@family}{\default@series}%
                    {\default@shape}}%
    \expandafter\let\csname#1-cmd\endcsname\@changed@cmd %
    \begingroup %
      \wlog{Now handling font encoding #1 ...}%
      \lowercase{%
        \InputIfFileExists{#1enc.dfu}}%
           {\wlog{... processing UTF-8 mapping file for font encoding #1}}%
           {\wlog{... no UTF-8 mapping file for font encoding #1}}%
    \endgroup
  \else
     \@font@info{Redeclaring font encoding #1}%
  \fi
  \global\@namedef{T@#1}{#2}%
  \global\@namedef{M@#1}{\default@M#3}%
  \xdef\LastDeclaredEncoding{#1}%
}

\DeclareUnicodeCharacter{00A9}{\textcopyright}
\DeclareUnicodeCharacter{00AA}{\textordfeminine}
\DeclareUnicodeCharacter{00AE}{\textregistered}
\DeclareUnicodeCharacter{00BA}{\textordmasculine}
\DeclareUnicodeCharacter{02C6}{\textasciicircum}
\DeclareUnicodeCharacter{02DC}{\textasciitilde}
\DeclareUnicodeCharacter{200C}{\textcompwordmark}
\DeclareUnicodeCharacter{2026}{\textellipsis}
\DeclareUnicodeCharacter{2122}{\texttrademark}
\DeclareUnicodeCharacter{2423}{\textvisiblespace}

%    \end{macrocode}
%
% \iffalse
%</def>
% \fi
%
% \iffalse
%<*defx>
% \fi
%
% \subsection{\texttt{lutf8x.def}}
%
%    \begin{macrocode}
%% This file was adapted from utf8.def, which copyright is:
%% Copyright 1993 1994 1995 1996 1997 1998 1999 2000 2001 2002 2003
%% 2004 2005 2006 The LaTeX3 Project.
%%
%% utf8.def is under the lppl version 1.3c or later, and can be found
%% in the base LaTeX system.
%%
%% The lppl can be found at http://www.latex-project.org/lppl.txt
%%
%% The changes to utf8.def are Copyright 2009 Elie Roux, and are under
%% the CC0 license.
%%
%% The changes are LuaTeX support.
%%
%% This file is distributed under the CC0 license, with clause 6 of the
%% lppl as additional restrictions.

%    \end{macrocode}
%
%    This file is mostly the code from \texttt{lutf.def}, but it adds
%    mechanisms to pass from \emph{legacy mode} to \emph{unicode font mode}.
%    The trick is to put in a lua table all characters that are activated by
%    the \emph{legacy mode}, and to unactivate them when we switch to
%    \emph{unicode font mode}. This is made (almost) entirely in lua. The
%    difficult part is the changes in |\DeclareFontEncoding|.
%
%    \begin{macrocode}

\ProvidesFile{lutf8x.def}
   [2010/05/10 v0.97 UTF-8 support for luainputenc]

\makeatletter
\catcode`\ \saved@space@catcode

\@inpenc@test

\ifx\@begindocumenthook\@undefined
  \makeatother
  \endinput \fi

%    \end{macrocode}
%
%    We change it a little to add the activated character in the lua table.
%
%    \begin{macrocode}

\gdef\DeclareUnicodeCharacter#1#2{%
 \@tempcnta"#1%
 \luadirect{luainputenc.declare_character('\the\@tempcnta')}%
 \catcode\@tempcnta\active %
 \DeclareInputText{\the\@tempcnta}{#2}%
}

\@onlypreamble\DeclareUnicodeCharacter

\def\cdp@elt#1#2#3#4{%
  \wlog{Now handling font encoding #1 ...}%
  \lowercase{%
      \InputIfFileExists{#1enc.dfu}}%
         {\wlog{... processing UTF-8 mapping file for font encoding
                #1}%
          \catcode`\ 9\relax}%
        {\wlog{... no UTF-8 mapping file for font encoding #1}}%
}
\cdp@list

%    \end{macrocode}
%
%    The macros to change from/to \emph{legacy mode} to/from \emph{unicode
%    font mode}.
%
%    \begin{macrocode}

\def\lIE@ActivateUnicodeCatcodes{%
\luadirect{luainputenc.activate_characters()}%
}

\def\lIE@DesactivateUnicodeCatcodes{%
\luadirect{luainputenc.desactivate_characters()}%
}

\def\lIE@CharactersActivated{%
\luadirect{luainputenc.force_characters_activated()}
}

\edef\lIE@EU{EU2}

%    \end{macrocode}
%
%    We add some code to automatically activate or unactivate characters
%    according to the encoding changes. Note that we override |\@@enc@update|,
%    which may pose some problems if a package of yours does it too.
%    Fortunately this package is the only one that does it in \TeX Live.
%
%    \begin{macrocode}

\def\DeclareFontEncoding@#1#2#3{%
  \edef\lIE@test{#1}%
  \ifx\lIE@test\lIE@EU %
    \ifx\LastDeclaredEncoding\lIE@EU\else %
      \lIE@CharactersActivated %
      \lIE@DesactivateUnicodeCatcodes %
    \fi
    \gdef\@@enc@update{%
      \edef\lIE@test{#1}%
      \ifx\f@encoding\lIE@EU %
        \lIE@DesactivateUnicodeCatcodes %
       \else %
        \lIE@ActivateUnicodeCatcodes %
      \fi
      \expandafter\let\csname\cf@encoding-cmd\endcsname\@changed@cmd
      \expandafter\let\csname\f@encoding-cmd\endcsname\@current@cmd
      \default@T
      \csname T@\f@encoding\endcsname
      \csname D@\f@encoding\endcsname
      \let\enc@update\relax
      \let\cf@encoding\f@encoding
    }
  \else %
    \expandafter %
    \ifx\csname T@#1\endcsname\relax %
      \def\cdp@elt{\noexpand\cdp@elt}%
      \xdef\cdp@list{\cdp@list\cdp@elt{#1}%
                      {\default@family}{\default@series}%
                      {\default@shape}}%
      \expandafter\let\csname#1-cmd\endcsname\@changed@cmd %
      \begingroup %
        \wlog{Now handling font encoding #1 ...}%
        \lowercase{%
          \InputIfFileExists{#1enc.dfu}}%
             {\wlog{... processing UTF-8 mapping file for font encoding #1}}%
             {\wlog{... no UTF-8 mapping file for font encoding #1}}%
      \endgroup
    \else
       \@font@info{Redeclaring font encoding #1}%
    \fi
  \fi %
  \global\@namedef{T@#1}{#2}%
  \global\@namedef{M@#1}{\default@M#3}%
  \xdef\LastDeclaredEncoding{#1}%
}

\DeclareUnicodeCharacter{00A9}{\textcopyright}
\DeclareUnicodeCharacter{00AA}{\textordfeminine}
\DeclareUnicodeCharacter{00AE}{\textregistered}
\DeclareUnicodeCharacter{00BA}{\textordmasculine}
\DeclareUnicodeCharacter{02C6}{\textasciicircum}
\DeclareUnicodeCharacter{02DC}{\textasciitilde}
\DeclareUnicodeCharacter{200C}{\textcompwordmark}
\DeclareUnicodeCharacter{2026}{\textellipsis}
\DeclareUnicodeCharacter{2122}{\texttrademark}
\DeclareUnicodeCharacter{2423}{\textvisiblespace}

%    \end{macrocode}
%
% \iffalse
%</defx>
% \fi
%
%
% \subsection{\texttt{luainputenc.lua}}
%
%    First the \texttt{inputenc} module is registered as a Lua\TeX\ module,
%    with some informations.
%
% \iffalse
%<*lua>
% \fi
%
%    \begin{macrocode}

module('luainputenc', package.seeall)

luainputenc.module = {
    name          = "luainputenc",
    version       =  0.97,
    date          = "2010/05/10",
    description   = "Lua simple inputenc package.",
    author        = "Elie Roux",
    copyright     = "Elie Roux",
    license       = "CC0",
}

luatexbase.provides_module(luainputenc.module)

local format = string.format

luainputenc.log = luainputenc.log or function(...)
  luatexbase.module_log('luainputenc', format(...))
end

%    \end{macrocode}
%
%    We keep the option and the true encoding in two variables.
%
%    \begin{macrocode}

luainputenc.encoding = "utf8"
luainputenc.package_option = nil

function luainputenc.set_option(option)
  luainputenc.package_option = option
  if option == "lutf8" or option == "lutf8x" or option == "utf8x" or option == "unactivate" then
    luainputenc.encoding = "utf8"
  else
    luainputenc.encoding = option
  end
end

%    \end{macrocode}
%
%    Some local declarations.
%
%    \begin{macrocode}

local char, utfchar, byte, format, gsub, utfbyte, utfgsub =
string.char, unicode.utf8.char, string.byte, string.format, string.gsub, unicode.utf8.byte, unicode.utf8.gsub

%    \end{macrocode}
%
%    The function to transform a 8-bit character in the corresponding fake
%    UTF-8 character.
%
%    \begin{macrocode}

function luainputenc.byte_to_utf(ch)
    return utfchar(byte(ch))
end

%    \end{macrocode}
%
%    The function that will be registered in the
%    \texttt{process\_input\_buffer} callback when needed.
%
%    \begin{macrocode}

function luainputenc.fake_utf_read(buf)
    return gsub(buf,"(.)", luainputenc.byte_to_utf)
end

%    \end{macrocode}
%
%    The function to transform a fake utf8 character in the corresponding 
%    8-bit character.
%
%    \begin{macrocode}

function luainputenc.utf_to_byte(ch)
    return char(utfbyte(ch))
end

%    \end{macrocode}
%
%    The function that will be registered in the
%    \texttt{process\_output\_buffer} callback if it exists.
%
%    \begin{macrocode}

function luainputenc.fake_utf_write(buf)
    return utfgsub(buf,"(.)", luainputenc.utf_to_byte)
end

%    \end{macrocode}
%
%    Here we register the two callbacks, and the behaviour is the same as in pdfTeX.
%    The next part of the file is only the machinery for LuaTeX versions without
%    the callback |process_output_buffer|, so it will be deprecated after
%    TeXLive 2009, you are not advised to use it.
%
%    \begin{macrocode}

if tex.luatexversion > 42 then

    function luainputenc.register_callbacks()
        luatexbase.add_to_callback('process_output_buffer', luainputenc.fake_utf_write, 'luainputenc.fake_utf_write')
        luatexbase.add_to_callback('process_input_buffer', luainputenc.fake_utf_read, 'luainputenc.fake_utf_read')
    end
    
else

%    \end{macrocode}
%
%    \texttt{start()} and \texttt{stop()} are the functions that register or
%    unregister the function in the callback. When the function is registered,
%    Lua\TeX\ reads the input in fake UTF-8.
%
%    \begin{macrocode}

    local started, stopped = 1, 0

    luainputenc.state = stopped

    function luainputenc.setstate(state)
        if state == luainputenc.state then
            return
        elseif state == started then
            luainputenc.start()
        else
            luainputenc.stop()
        end
    end

    function luainputenc.setstarted()
        luainputenc.setstate(started)
    end

    function luainputenc.setstopped()
        luainputenc.setstate(stopped)
    end

    function luainputenc.start()
        luatexbase.add_to_callback('process_input_buffer', luainputenc.fake_utf_read,
            'luainputenc.fake_utf_read')
        luainputenc.state = started
        if luainputenc.callback_registered == 0 then
            luainputenc.register_callback()
        end
    end
    
    function luainputenc.stop()
        luatexbase.remove_from_callback('process_input_buffer', 'luainputenc.fake_utf_read')
        luainputenc.state = stopped
        return
    end

%    \end{macrocode}
%
%    Here is a list of all file extentions for which we consider that the
%    files have been written by Lua\TeX , and thus must be read in fake UTF-8.
%    I may have forgotten things in the list. If you find a new extention,
%    please report the maintainer.
%
%    \begin{macrocode}

    luainputenc.unicode_extentions = {
      ['.aux'] = 1, -- basic files
      ['.toc'] = 1,
      ['.gls'] = 1,
      ['.ind'] = 1,
      ['.idx'] = 1,
      ['.vrb'] = 1, -- beamer and powerdot
      ['.nav'] = 1, -- other beamer extentions
      ['.sol'] = 1,
      ['.qsl'] = 1,
      ['.snm'] = 1,
      ['.pgn'] = 1, -- pagereference
      ['.cpg'] = 1, -- AlProTeX
      ['.pst'] = 1, -- pst-tree
      ['.tmp'] = 1, -- sauerj/collect
      ['.sym'] = 1, -- listofsymbols
      ['.sub'] = 1, -- listofsymbols
      ['.lof'] = 1, -- preprint
      ['.lot'] = 1, -- preprint
      ['mtc1'] = 1, -- minitoc
      ['.ovr'] = 1, -- thumbss
      ['.fff'] = 1, -- endplate
      ['.sbb'] = 1, -- splitbib
      ['.bbl'] = 1, -- latex
      ['.ain'] = 1, -- authorindex
      ['.abb'] = 1, -- juraabbrev
      ['.ent'] = 1, -- endnotes
      ['.end'] = 1, -- fn2end
      ['.thm'] = 1, -- ntheorem
      ['.xtr'] = 1, -- extract
      ['.han'] = 1, -- linguho
      ['.bnd'] = 1, -- bibref
      ['.bbl'] = 1, -- bibref
      ['.col'] = 1, -- mwrite
      ['.ttt'] = 1, -- endfloat
      ['.fax'] = 1, -- lettre
      ['.tns'] = 1, -- lettre
      ['.odt'] = 1, -- lettre
      ['.etq'] = 1, -- lettre
      ['.emd'] = 1, -- poemscol
      ['.emx'] = 1, -- poemscol
      ['.ctn'] = 1, -- poemscol
      ['.hst'] = 1, -- vhistory
      ['.acr'] = 1, -- crosswrd
      ['.dwn'] = 1, -- crosswrd
      ['.ttc'] = 1, -- talk
      -- ['.txt'] = 1, -- coverpage, but not sure it's safe to include it...
      ['.eve'] = 1, -- calend0
      ['.scn'] = 1, -- cwebmac
      }

%    \end{macrocode}
%
%    The code to define a specific behaviour for certain files.
%
%    \begin{macrocode}

    luainputenc.unicode_files = {}

    luainputenc.non_unicode_files = {}

    function luainputenc.set_unicode_file(filename)
        if luainputenc.non_unicode_files[filename] == 1 then
            luainputenc.non_unicode_files[filename] = nil
        end
        luainputenc.unicode_files[filename] = 1
    end

    function luainputenc.set_non_unicode_file(filename)
        if luainputenc.unicode_files[filename] == 1 then
            luainputenc.unicode_files[filename] = nil
        end
        luainputenc.non_unicode_files[filename] = 1
    end

    function luainputenc.set_unicode_extention(ext)
        luainputenc.unicode_extention[ext] = 1
    end

    function luainputenc.set_non_unicode_extention(ext)
        if luainputenc.unicode_extentions[ext] == 1 then
            luainputenc.unicode_extentions[ext] = nil
        end
    end

    function luainputenc.unset_file(filename)
        if luainputenc.unicode_files[filename] == 1 then
            luainputenc.unicode_files[filename] = nil
        elseif luainputenc.non_unicode_files[filename] == 1 then
            luainputenc.non_unicode_files[filename] = nil
        end
    end

    local unicode, non_unicode = stopped, started

    function luainputenc.find_state(filename)
        if luainputenc.unicode_files[filename] == 1 then
            return unicode
        elseif luainputenc.non_unicode_files[filename] == 1 then
            return non_unicode
        else
            local ext = filename:sub(-4)
             if luainputenc.unicode_extentions[ext] == 1 then
                 return unicode
             else
                return non_unicode
             end
        end
    end

%    \end{macrocode}
%
%    We register the functions to stop or start the fake UTF-8 translation in
%    the appropriate callbacks if necessary.
%
%    \begin{macrocode}

    function luainputenc.pre_read_file(env)
        if not env.path then 
            return
        end
        local currentstate = luainputenc.state
        luainputenc.setstate(luainputenc.find_state(env.filename))
        env.previousstate = currentstate
    end

    function luainputenc.close(env)
        luainputenc.setstate(env.previousstate)
    end

    luainputenc.callback_registered = 0

    function luainputenc.register_callback()
        if luainputenc.callback_registered == 0 then
            luatexbase.add_to_callback('pre_read_file', luainputenc.pre_read_file,
                'luainputenc.pre_read_file')
            luatexbase.add_to_callback('file_close', luainputenc.close, 'luainputenc.close')
            luainputenc.callback_registered = 1
        end
    end

end

%    \end{macrocode}
%
%    Finally we provide some functions to activate or disactivate the catcodes
%    of the non-ASCII characters.
%
%    \begin{macrocode}


luainputenc.activated_characters = {}
luainputenc.characters_are_activated = false

function luainputenc.declare_character(c)
    luainputenc.activated_characters[tonumber(c)] = true
end

function luainputenc.force_characters_activated ()
    luainputenc.characters_are_activated = true
end

function luainputenc.activate_characters()
    if not luainputenc.characters_are_activated then
        for n, _ in pairs(luainputenc.activated_characters) do
            tex.sprint(string.format('\\catcode %d\\active',n))
        end
        luainputenc.characters_are_activated = true
    end
end

function luainputenc.desactivate_characters()
    if luainputenc.characters_are_activated then
        for n, _ in pairs(luainputenc.activated_characters) do
            tex.sprint(string.format('\\catcode %d=11',n))
        end
        luainputenc.characters_are_activated = false
    end
end

%    \end{macrocode}
%
% \iffalse
%</lua>
% \fi
%
% \Finale
\endinput
